\documentclass{article}
\usepackage[utf8]{inputenc}

\title{Scientific Computing - Exercise Sheet 2}
\author{Jonathan Hellwig, Jule Schütt, Mika Tode, Giuliano Taccogna}
\date{22.04.2021}

\usepackage{natbib}
\usepackage{graphicx}
\usepackage{algorithmicx}
\usepackage{algpseudocode}
\usepackage{enumitem}
\usepackage{amssymb}
\usepackage{amsmath}

\begin{document}

\maketitle

\section{Exercise}
\begin{enumerate}[label=(\alph*)]
  \item Computation for $t_{total}^\nu$, where $\nu=0.01$ and $p=10^k$ for $k\in\{1,\dots,5\}$:
    \begin{align*}
      t^{\nu}_{total}   &= \nu t_1^N+(1-\nu)t_p^N\\
                        &= 10^{-2} \cdot 10^5h + (1 - 10^{-2}) \cdot 10^5h\cdot 10^{-k} \\
                        &= 1000h + 0.99 \cdot 10^{5-k}h
    \end{align*}
    \begin{center}
      Values for different $k$:\\~\\
      \begin{tabular}{ l | c } 
        $k$ & $t^{\nu}_{total}$ \\
        \hline
        1 & 10900h \\
        2 & 1990h  \\
        3 & 1099h  \\
        4 & 1009.9h  \\
        5 & 1000.99h  \\
      \end{tabular}        
    \end{center}

  \item The theoretical speedup and the theoretical efficiency are the following values:
    \begin{align*}
      S_p^T &= \frac{t_1^N}{t_p^N} = 10^k,\qquad \text{for }k\in\{1,\dots,5\}\\
      E_p^T &= \frac{t_1^N}{pt_p^N} = 1,
    \end{align*}
    where we still use $p=10^k$ for $k\in\{1,\dots,5\}.$\\
    We have to exchange the theoretical computing time $t_p^N$ where we assumed that every part of the algorithm can be parallelized with the real computing time $t_{total}^N$ which include that we can't parallelize just $99\%$ of the algorithm.
    Therefore we have a look at the effective speedup $S_p^E$ and efficiency $E_p^E$. It holds
    \begin{align*}
      S_p^E &= \frac{t_1^N}{t_{total}^\nu}= \frac{10^5}{t_{total}^\nu} = \frac{10^5}{1000h + 0.99 \cdot 10^{5-k}h}\\
      E_p^E &= \frac{t_1^N}{pt_{total}^\nu} = \frac{10^5}{pt_{total}^\nu} = \frac{10^5}{p(1000h + 0.99 \cdot 10^{5-k}h)}
    \end{align*}
    \begin{center}
      Values for different $k$:\\~\\
      \begin{tabular}{ l | c | c } 
        $k$ & $S_p$ & $E_p$\\
        \hline
        1 &  9,17 & 0.917 \\
        2 &  50.25 & 0.5025 \\
        3 &  90.99 & 0.09099 \\
        4 &  99.02 & 0.009902 \\
        5 &  99.9 & 0.000999 \\
      \end{tabular}        
    \end{center}

  \item It holds: 
    \begin{align*}
      t^{\nu}_{total} = \nu t_1^N + (1-\nu)t_p^N = t_s^N + (1-\nu)\underbrace{\frac{t_1^N}{p}}_{\to 0} \xrightarrow[p \to \infty]{} t_s^N=1000.
    \end{align*}
    Using that we compute the limit of the speedup and the efficiency:
    \begin{align*}
      S_p^E &= \frac{t_1^N}{t_{total}^\nu}\xrightarrow[p \to \infty]{} \frac{t_1^N}{t_s^N}=\frac{1}{\nu}=100\\
      E_p^E &= \frac{t_1^N}{pt_{total}^\nu}\xrightarrow[p \to \infty]{}0,
    \end{align*}
    where the last limit results from the convergence of $t^\nu_{total}$ to a positive number such that $pt^\nu_{total}\to\infty$ when $p\to\infty$.
  \item Now the communication time overhead $t_c$ is increasing by 1 when $p$ increases by $10$. Since we look on $p=10^k$, we can identify $t_c=\frac{p}{10}$.
    It follows
    \begin{align*}
      S^c_p = \frac{t_1^N}{t_p^N+t_c} = \frac{10^5}{\frac{10^5}{p} + \frac{p}{10}} \xrightarrow[k \to \infty]{} 0,
    \end{align*}
    where $t_1^N = 10^5$.\\
    Now we look at the speedup while insert values $p=10^k$, $k\in\{1,\dots,5\}$:
    \begin{center}
      \begin{tabular}{ l | c } 
        $k$ & $S_p^c$ \\
        \hline
        1 & 9.99 \\
        2 & 99.01 \\
        3 & 500 \\
        4 & 99.01 \\
        5 & 9.99 \\
      \end{tabular}        
    \end{center}
    \begin{figure}[hbt!]
      \centering
      \includegraphics[width=8cm]{speedup.png}
      \caption{Effective speedup depending on the number of rings}
    \end{figure}
\end{enumerate}
\end{document}
