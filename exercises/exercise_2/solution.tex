\documentclass{article}
\usepackage[utf8]{inputenc}

\title{Scientific Computing - Exercise Sheet 2}
\author{Jonathan Hellwig, Jule Schütt, Mika Tode, Giuliano Taccogna}
\date{19.04.2021}

\usepackage{natbib}
\usepackage{graphicx}
\usepackage{algorithmicx}
\usepackage{algpseudocode}
\usepackage{enumitem}
\usepackage{amssymb}
\usepackage{amsmath}

\begin{document}

\maketitle

\section{Exercise}
\begin{enumerate}[label=(\alph*)]
  \item 
    \begin{align*}
      t^{\nu}_{total} &= 10^{-2} \cdot 10^5h + (1 - 10^{-2}) \cdot 10^5h\cdot 10^{-k} \\
                      &= 1000h + 0.99 \cdot 10^{5-k}h
    \end{align*}
    \begin{tabular}{ l | c } 
  k & $t^{\nu}_{total}$ \\
  \hline
  1 & 10900h \\
  2 & 1990h  \\
  3 & 1099h  \\
  4 & 1009.9h  \\
  5 & 1000.99h  \\
\end{tabular}
\item $S_p = \frac{t_1^N}{t_p^N} = 10^k$ \\
  $E_p = \frac{t_1^N}{pt_p^N} = 1$
\item $t^{\nu}_{total} = \nu t_1^N + (1-\nu)t_p^N = t_s^N + (1-\nu)\frac{t_1^N}{p} \xrightarrow[p \to \infty]{} t_s^N$
\item $S^c_p = \frac{t_1^N}{t_p^N+t_c} = \frac{t_1^N}{\frac{t_1^N}{10k} + 10k} \xrightarrow[k \to \infty]{} 0$
\end{enumerate}
\end{document}
