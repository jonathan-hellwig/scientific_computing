\documentclass{article}
\usepackage[utf8]{inputenc}

\title{Scientific Computing - Exercise Sheet 1}
\author{Jonathan Hellwig, Jule Schütt, Mika Tode, Giuliano Taccogna}
\date{19.04.2021}

\usepackage{float}
\usepackage{svg}
\usepackage{natbib}
\usepackage{graphicx}
\usepackage{algorithmicx}
\usepackage{algpseudocode}
\usepackage{enumitem}
\usepackage{amssymb}

\begin{document}

\maketitle

\section{Exercise}
\begin{enumerate}[label=(\alph*)]
  \item \textbf{Input}: $ \textbf{A} \in \mathbb{R}^{n\times n} \quad b, x_0\in \mathbb{R}^n$
    \begin{algorithmic}[1]
	\State $h_{0} = \textbf{A}x_{0}$ (matrix multiplication)
	\State $r_0 = b - h_0$ (1.1 loop)
	\State $p_0 = r_0$
	\For{$k = 1,2,\dots$}
	\State $h_{k-1} = \textbf{A}p_{k-1}$ (matrix multiplication)
	\State $\gamma_{k-1} = p^{T}_{k-1}\cdot h_{k-1}$ (2.1 loop)
	\State $\beta_{k-1} = r^{T}_{k-1}\cdot r_{k-1}$ (2.2 loop)
	\State $\delta_{k-1} = r^{T}_{k-1}\cdot h_{k-1}$ (2.3 loop)
	\State $\zeta_{k-1} = h^{T}_{k-1}\cdot h_{k-1}$ (2.4 loop)
	\State $\alpha_{k-1} = \frac{\beta^{k-1}}{\gamma^{k-1}}$
	\State $\beta_{k} = \beta_{k-1} - 2 \alpha_{k-1} \delta_{k-1} + \alpha_{k-1}^2\zeta_{k-1}$
	\State $x_k = x_{k-1} + \alpha_{k-1}p_{k-1}$ (2.5 loop)
	\State $r_k = r_{k-1} - \alpha_{k-1}h_{k-1}$ (2.6 loop)
	\State $p_k = r_{k} + \frac{\beta_k}{\beta_{k-1}}p_{k-1}$ (2.7 loop)
	\EndFor
    \end{algorithmic}
All sections which are marked to be a loop can be parallelized. The minimal number of loops within the iteration loop is 2, since the loops
2.1, 2.2, 2.3, 2.4 and the loops 2.5, 2.6, 2.7 can be summarized to one loop respectively.

  \item \textbf{Input}: $ \textbf{A} \in \mathbb{R}^{n\times n} \quad b, x_0\in \mathbb{R}^n$
    \begin{algorithmic}[1]
	\State $h_{0} = \textbf{A}x_{0}$ (matrix multiplication)
	\State \textbf{begin parallel private$(i, r_{0}$) shared$(b, h_{0})$}
	\State $r_0 = b - h_0$ (1.1 loop)
	\State \textbf{end parallel}
	\State $p_0 = r_0$
	\For{$k = 1,2,\dots$}
	\State $h_{k-1} = \textbf{A}p_{k-1}$ (matrix multiplication)
	\State \textbf{begin parallel private$(i, \gamma_{(k-1)}, \beta_{(k-1)}, \delta_{(k-1)}, \zeta_{(k-1)})$ shared$(r_{k-1},p_{k-1},h_{k-1})$}
	\State $\gamma_{k-1} = p^{T}_{k-1}\cdot h_{k-1}$ (2.1 loop)
	\State $\beta_{k-1} = r^{T}_{k-1}\cdot r_{k-1}$ (2.2 loop)
	\State $\delta_{k-1} = r^{T}_{k-1}\cdot h_{k-1}$ (2.3 loop)
	\State $\zeta_{k-1} = h^{T}_{k-1}\cdot h_{k-1}$ (2.4 loop)
	\State \textbf{end parallel}
	\State $\alpha_{k-1} = \frac{\beta^{k-1}}{\gamma^{k-1}}$
	\State $\beta_{k} = \beta_{k-1} - 2 \alpha_{k-1} \delta_{k-1} + \alpha_{k-1}^2\zeta_{k-1}$
	\State \textbf{begin parallel private$(i, x_{k}, r_{k}, p_{k})$ shared$(x_{k-1}, \alpha_{k-1}, p_{k-1}, r_{k-1}, h_{k-1},  r_{k},  \beta_{k}, \beta_{k-1})$}
	\State $x_k = x_{k-1} + \alpha_{k-1}p_{k-1}$ (2.5 loop)
	\State $r_k = r_{k-1} - \alpha_{k-1}h_{k-1}$ (2.6 loop)
	\State $p_k = r_{k} + \frac{\beta_k}{\beta_{k-1}}p_{k-1}$ (2.7 loop)
 	\State \textbf{end parallel}
	\EndFor
    \end{algorithmic}
Here $i$ denotes the counter, which is needed for the loop operation. Notice that for the Vectormultiplications in the loops 2.1, 2.2, 2.3 and 2.4 normally also a \textbf{reduce} operation should be included.

  \item \textbf{Input}: $ \textbf{A} \in \mathbb{R}^{n\times n} \quad b, x_0\in \mathbb{R}^n$
    \begin{algorithmic}[1]
	\State $h_{0} = \textbf{A}x_{0}$ (matrix multiplication)
	\State $r_0^{(0)} = b^{(0)} - h_0^{(0)}$ (1.1 loop)
	\State $r_0^{(1)} = b^{(1)} - h_0^{(1)}$ (1.1 loop)
	\State \textbf{barrier}
	\State \textbf{send}($r_0^{(0)}$)
	\State \textbf{recv}($r_0^{(0)}$)
	\State \textbf{send}($r_0^{(1)}$)
	\State \textbf{recv}($r_0^{(1)}$)
	\State $p_0 = r_0$
	\For{$k = 1,2,\dots$}
	\State $h_{k-1} = \textbf{A}p_{k-1}$ (matrix multiplication)
	\State $\gamma_{k-1}^{(0)} = p^{T(0)}_{k-1}\cdot h_{k-1}^{(0)}$ (2.1 loop)
	\State $\gamma_{k-1}^{(1)} = p^{T(1)}_{k-1}\cdot h_{k-1}^{(1)}$ (2.1 loop)
	\State $\beta_{k-1}^{(0)} = r^{T(0)}_{k-1}\cdot r_{k-1}^{(0)}$ (2.2 loop)
	\State $\beta_{k-1}^{(1)} = r^{T(1)}_{k-1}\cdot r_{k-1}^{(1)}$ (2.2 loop)
	\State $\delta_{k-1}^{(0)} = r^{T(0)}_{k-1}\cdot h_{k-1}^{(0)}$ (2.3 loop)
	\State $\delta_{k-1}^{(1)} = r^{T(1)}_{k-1}\cdot h_{k-1}^{(1)}$ (2.3 loop)
	\State $\zeta_{k-1}^{(0)} = h^{T(0)}_{k-1}\cdot h_{k-1}^{(0)}$ (2.4 loop)
	\State $\zeta_{k-1}^{(1)} = h^{T(1)}_{k-1}\cdot h_{k-1}^{(1)}$ (2.4 loop)
	\State \textbf{barrier}
      	\State \textbf{send($\gamma_{k-1}^{(0)}, \beta_{k-1}^{(0)}, \delta_{k-1}^{(0)}, \zeta_{k-1}^{(0)}$)}
      	\State \textbf{recv($\gamma_{k-1}^{(0)}, \beta_{k-1}^{(0)}, \delta_{k-1}^{(0)}, \zeta_{k-1}^{(0)}$)}
	\State \textbf{send($\gamma_{k-1}^{(1)}, \beta_{k-1}^{(1)}, \delta_{k-1}^{(1)}, \zeta_{k-1}^{(1)}$)}
      	\State \textbf{recv($\gamma_{k-1}^{(1)}, \beta_{k-1}^{(1)}, \delta_{k-1}^{(1)}, \zeta_{k-1}^{(1)}$)}
	\State $\alpha_{k-1} = \frac{\beta^{k-1}}{\gamma^{k-1}}$
	\State $\beta_{k} = \beta_{k-1} - 2 \alpha_{k-1} \delta_{k-1} + \alpha_{k-1}^2\zeta_{k-1}$
	\State $x_k^{(0)} = x_{k-1}^{(0)} + \alpha_{k-1}p_{k-1}^{(0)}$ (2.5 loop)
	\State $x_k^{(1)} = x_{k-1}^{(1)}  + \alpha_{k-1}p_{k-1}^{(1)} $ (2.5 loop)
	\State $r_k^{(0)} = r_{k-1}^{(0)} - \alpha_{k-1}h_{k-1}^{(0)}$ (2.6 loop)
	\State $r_k^{(1)}  = r_{k-1}^{(1)}  - \alpha_{k-1}h_{k-1}^{(1)} $ (2.6 loop)
	\State $p_k^{(0)} = r_{k}^{(0)} + \frac{\beta_k}{\beta_{k-1}}p_{k-1}^{(0)}$ (2.7 loop)
	\State $p_k^{(1)}  = r_{k}^{(1)}  + \frac{\beta_k}{\beta_{k-1}}p_{k-1}^{(1)} $ (2.7 loop)
	\State \textbf{barrier}
	\State \textbf{send($p_{k}^{(0)} $)}
	\State \textbf{recv($p_{k}^{(0)} $)}
	\State \textbf{send($p_{k}^{(1)} $)}
	\State \textbf{recv($p_{k}^{(1)} $)}  	  	
	\EndFor
    \end{algorithmic}
We determine that for all parallelizable parts Processor 0 works on the for-loop for $i = 1,...,n/2$ and Processor 1 works on the for-loop for $i = n/2+1,...,n$, with $n/2$ possibly rounded.
Accordingly $v^{(j)}$ denotes a scalarproduct or a vector which is calculated by processor $j = 0, 1$. 
If $v^{(j)}$ denotes a Vector we mean $v^{(0)} = (v(1), v(2), ..., v(n/2))^T$ and $v^{(1)} = (v(n/2+1), v(n/2+2), ..., v(n))^T$.
If $v^{(j)}$ denotes a scalarproduct, we mean $v^{(j)} = H^{T(j)} \cdot K^{(j)}$ for $H, K \in \mathbb{R}^n$.
If a processor recives data through the \textbf{recv} command, we implicitly include in \textbf{recv} that the recieved data is combined with the data that was calculated on the processor itself in the right way. That means if one half of a vector is recieved \textbf{recv} creates the whole vector. If "one half" of a scalarproduct is recived \textbf{recv} adds the corresponding scalarproducts to the final scalarproduct.
If the processor is not specified in a calculation, the calculation is executed on both processors.
\end{enumerate} 
\end{document}